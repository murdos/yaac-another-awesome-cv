% Awesome Source CV LaTeX Template
%
% This template has been downloaded from:
% https://github.com/darwiin/awesome-neue-latex-cv
%
% Author:
% Christophe Roger
%
% Template license:
% CC BY-SA 4.0 (https://creativecommons.org/licenses/by-sa/4.0/)

%Section: Work Experience at the top
\sectionTitle{Expériences Professionelles}{\faSuitcase}
%\renewcommand{\labelitemi}{$\bullet$}
\begin{experiences}
  \experience
    {Aujourd'hui}   {Lead dev Java - Projet DECOFER}{Client : \link{https://www.rte-france.com}{Rte}}{ESN : \link{https://www.sully-group.fr}{Sully-Group}}
    {Août 2018} {
                      \begin{itemize}
                        \item Projet de développement en mode agile d'un système d'information permettant la télérelève des consommations électriques ferroviaires
                        \item Développements backend (gateway, microservice référentiel, microservice entrepôt de données, Flink)
                        \item Participation à la définition de l'architecture et à la conception DevOps
                        \item Ateliers de pré-conception (Product Backlog Refinement)
                        \item Développement d'un blueprint JHipster dédié au projet
                      \end{itemize}
                    }
                    {Scrum,TDD,JHipster,Java8,Spring Boot,Hibernate,Javers,Flink,PostgreSQL,RabbitMQ,Cassandra,Camunda,Maven,Jenkins}
  \emptySeparator
  \experience
    {Juillet 2018} {Chef de projet TMA | Référent technique}{Client : \link{https://www.franceagrimer.fr}{FranceAgriMer}}{ESN : \link{https://www.sully-group.fr}{Sully-Group}}
    {Décembre 2017}    {
                      \begin{itemize}
                        \item Support et encadrement technique des équipes de développement
                        \item Suivi, validation et intégration des développements externalisés
                        \item Implémentation, analyse et livraison de correctifs de bugs sur les applicatifs métiers
                        \item Evolutions et corrections des bugs du framework de développement interne
                        \item Rédaction des dossiers d'architecture en collaboration avec les architectes fonctionnels
                        \item Veille technologique
                      \end{itemize}
                    }
                    {IntelliJ Idea,JBoss EAP,Eclipse,Maven,Jenkins,Nexus}
  \emptySeparator
  \experience
    {Novembre 2017}     {Chef de projet technique - Application SIAMOA}{Client : \link{https://www.cerema.fr}{CEREMA}}{ESN : \link{https://www.sully-group.fr}{Sully-Group}}
    {Mai 2016}    {
                      \begin{itemize}
                        \item Application de gestion des ouvrages d'art du réseau routier national : inventaire, suivi des actions d'évaluation et de maintenance
                        \item Migration du serveur métier vers Open Cobol : suivi de projet et reporting
                        \item Solution documentaire collaborative (wiki) : mise en place et formation
                        \item Evolutions et corrections : analyse, conception et développement
                        \item Mise en place de conventions de code
                        \item Mise en place d'un framework de développement d'interface web (jQuery, Bootstrap, taglibs)
                      \end{itemize}
                    }
                    {Tomcat,Spring,Eclipse,Maven,Oracle DB,Hibernate,RichFaces,AngularJS,jQuery,Bootstrap,LESS}
  \emptySeparator
  \consultantexperience
  {Mars 2012}       {Ingénieur Consultant}{Altran Technologies}{France}
  {Décembre 2007}   {IT Specialist}{IBM, Software Solutions Center of Excellence}
                    {
                      \begin{itemize}
                        \item \textbf{Projet eTACT} pour \href{https://www.edqm.eu/fr/contexte-mission-cd-p-phcmed.html}{EDQM} : Conception et développement JEE.
                        \item Application \emph{Android} pour tablette : Conception et développement.
                        \item Projets d'intégration, \emph{Enterprise Service Bus} (ESB) et moteur de processus:
                        \begin{itemize}
                          \item Conception et développement JEE
                          \item Définition et implémentation des processus métiers et médiations
                        \end{itemize}
                        \item Solutions RFID : Conception et développement Java (JEE, JSE et JME), Analyse, \emph{POC}, documentation et présentation technique du protocole ONS
                      \end{itemize}
                    }
                    {Rational Software Architect (\emph{RSA}),Eclipse,\emph{WAS} 7,DB2,Hibernate,Ant,RichFaces,Infosphere Traceability Server,Android,Websphere Integration Developer,Websphere Process Server}
  \emptySeparator
  \experience
  {Novembre 2007}  {Ingénieur d'étude}{IBM}{France}
  {Février 2007}   {
                      Projet de prototypage \emph{Campus Nova} pour le Crédit Agricole : Développement d'une solution de paiement NFC sur téléphones portables.
                      \begin{itemize}
                        \item Implémentation d'un porte monnaie électronique
                        \item Intégration avec une plateforme de paiement en ligne
                      \end{itemize}
                  }
                  {J2ME,Java Card,DB2,\emph{WAS}}
\end{experiences}
